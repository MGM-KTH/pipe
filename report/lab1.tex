%%%%%%%%%%%%%%%%%%%%%%%%%%%%%%%%%%%%%%%%%
% Short Sectioned Assignment
% LaTeX Template
% Version 1.0 (5/5/12)
%
% This template has been downloaded from:
% http://www.LaTeXTemplates.com
%
% Original author:
% Frits Wenneker (http://www.howtotex.com)
%
% License:
% CC BY-NC-SA 3.0 (http://creativecommons.org/licenses/by-nc-sa/3.0/)
%
%%%%%%%%%%%%%%%%%%%%%%%%%%%%%%%%%%%%%%%%%

%------------------------------------------------------------------------------
%	PACKAGES AND OTHER DOCUMENT CONFIGURATIONS
%------------------------------------------------------------------------------

\documentclass[paper=a4, fontsize=11pt]{scrartcl} % A4 paper and 11pt font size

\usepackage[T1]{fontenc} % Use 8-bit encoding that has 256 glyphs
%\usepackage{fourier} % Use the Adobe Utopia font for the document - comment this line to return to the LaTeX default
\usepackage[english]{babel} % English language/hyphenation
\usepackage{amsmath,amsfonts,amsthm} % Math packages

\usepackage[utf8]{inputenc} % Needed to support swedish "åäö" chars
\usepackage{titling} % Used to re-style maketitle
\usepackage{enumerate}
\usepackage{lipsum} % Used for inserting dummy 'Lorem ipsum' text into the template

\usepackage{hyperref}
\usepackage{url}

\usepackage{sectsty} % Allows customizing section commands
\allsectionsfont{\normalfont} % Make all sections centered, the default font and small caps

\usepackage{fancyhdr} % Custom headers and footers
\pagestyle{fancyplain} % Makes all pages in the document conform to the custom headers and footers
\fancyhead{} % No page header - if you want one, create it in the same way as the footers below
\fancyfoot[L]{} % Empty left footer
\fancyfoot[C]{} % Empty center footer
\fancyfoot[R]{\thepage} % Page numbering for right footer
\renewcommand{\headrulewidth}{0pt} % Remove header underlines
\renewcommand{\footrulewidth}{0pt} % Remove footer underlines
\setlength{\headheight}{13.6pt} % Customize the height of the header

\numberwithin{equation}{section} % Number equations within sections (i.e. 1.1, 1.2, 2.1, 2.2 instead of 1, 2, 3, 4)
\numberwithin{figure}{section} % Number figures within sections (i.e. 1.1, 1.2, 2.1, 2.2 instead of 1, 2, 3, 4)
\numberwithin{table}{section} % Number tables within sections (i.e. 1.1, 1.2, 2.1, 2.2 instead of 1, 2, 3, 4)

\setlength\parindent{0pt} % Removes all indentation from paragraphs - comment this line for an assignment with lots of text

\usepackage{fancyvrb}
\DefineShortVerb{\|}


\posttitle{\par\end{center}} % Remove space between author and title
%----------------------------------------------------------------------------------------
% TITLE SECTION
%----------------------------------------------------------------------------------------

\title{ 
\huge Laboration 1 \\ Digenv - processkommunikation med pipes \\ % The assignment title
\vspace{10pt}
\normalfont \normalsize 
\textsc{ID2200 - Operating Systems } \\ [25pt] %
}

\author{Gustaf Lindstedt \\ glindste@kth.se \\ 910301-2135 \and Martin Runelöv \\ mrunelov@kth.se \\ 900330-5738}

\date{\vspace{8pt}\normalsize\today} % Today's date or a custom date

\begin{document}

\maketitle

\section{Inledning}

% OBS! Se till så att filerna är läsbara för assistenterna (se 3.4 ).
Koden finns tillgänglig i katalogen |/afs/nada.kth.se/home/6/u1q11n66/os/lab1|.\\


% Verksamhetsberättelse och synpunkter på laborationen. Beskriv arbetets utveckling. Hade du problem med verktygen,
% kompilatorn m.fl.? Hur lång tid har arbetet tagit? Skriv dina förslag till förändringar, idéer etc. Tyck fritt! 
% Vi är angelägna om att få respons, så att vi kan förbättra till nästa år.

% Uppskattning av tidsåtgång och eventuella kommentarer kring laborationen
\subsection{Verksamhetsberättelse}
Det största problemet var val av programstruktur. 
Det finns flera olika bra sätt att genomföra laborationen på och att välja ett smidigt tillvägagångssätt redan från början visade sig vara svårt. Vårt slutgiltiga designval visade sig dessutom vara bristfälligt när det kommer till process-specific felhantering. I övrigt gick arbetet bra. Hantering av pipes gick smidigt. Vi hade ett problem där exit-status från children via wait gav oväntade värden. Detta löstes med macrot |WEXITSTATUS| (från |sys/wait.h|).


\subsection{Synpunkter}
PDFen ''Användbara systemanrop och biblioteksfunktioner'' är väldigt välskriven och hjälpte väldigt mycket.
Labben gick snabbare än väntat, totalt cirka 10 timmar. Dock har vi båda viss C-vana.\\

Det största problemet var den stora förvirringen kring krav och instruktioner. 
Till att börja med finns det rapportinstruktioner på två ställen, i labPM samt labbspecifika instruktioner i labbpeket. 
Det skulle vara skönt att ha en heltäckande checklista i labbpeket istället. \\

Labbpeket säger att man ska ta bort filer med |unlink|. Gäller detta fortfarande? Vi brukar använda |rm|.\\
Ni nämner att |//|-kommentarer inte fungerar i C. De flesta moderna C-kompilatorer tillåter dessa (C99). Vi vet att all kod ska skrivas i ANSI-C, men detta borde kanske förtydligas i labbpeket. Man borde dessutom kanske lägga till någon ANSI-flagga i kompileringen också, t.ex. |-std=C90| eller |-ansi|.\\

Ni hänvisar till filen CDOK som innehåller information om kommentarer i C. Filen ligger inte på kurshemsidan (KTH Social). Vi hittade den istället på den gamla kurshemsidan. Denna är dessutom väldigt heltäckande, vilket är bra generellt, men det skulle vara skönt att ha någon kort sammanfattning av denna som tar med det som är relevant för labben.


\section{Förberedelsefrågor}
\begin{enumerate}[1)]
\item \emph{När en maskin bootar med UNIX skapas en process som har PID=1 och
  den lever så länge maskinen är uppe. Från den här processen skapas alla andra
processer med fork. Vad heter denna process?}

-- Processen heter \emph{init} på Linux, \emph{launchd} på Mac.

\item \emph{Kan environmentvariabler användas för att kommunicera mellan
  föräldra- och barnprocess? åt bägge hållen?}

-- Child-processer får en kopia av parent-processens environmentvariabler.
Detta innebär att en parent kan bestämma startvärdena för children.
Eftersom child-processen får en kopia speglas inte ändringar i denna i
parent-processens variabler. Alltså kan inte child-processer kommunicera till
parent via dessa.

\item \emph{Man kan tänka sig att skapa en odödlig child-process som fångar
  alla SIGKILL-signaler genom att registrera en egen signalhanterare
kill\_handler som bara struntar i SIGKILL. Processen ska förstås ligga i en
oändlig loop då den inte exekverar signalhanteraren. Testa! Skriv ett program
med en sådan signalhanterare, kompilera och provkör. Vad händer?}

-- Se appendix för programmet, |immortal.c| \\
-- Det är inte tillåtet att fånga SIGKILL då detta skulle vara en
säkerhetsrisk. Det går däremot att fånga SIGINT.
Resultat av ett program som försöker registrera en signalhanterare för SIGKILL:

\begin{verbatim}
sigaction() failed: Invalid argument
\end{verbatim}

Förklaringen finns i man-filen för sigaction:
\begin{verbatim}
ERRORS
     The sigaction() system call will fail and no new signal handler will be
     installed if one of the following occurs:

     [EINVAL]           An attempt is made to ignore or supply a handler for
                        SIGKILL or SIGSTOP.
\end{verbatim}

\item \emph{Varför returnerar fork 0 till child-processen och child-PID till
  parent-processen, i stället för tvärtom?}

-- Parent-Children är en one-to-many-relation. En child-process kan anropa
getppid() för att få parent-processens PID,
men det enda sättet för en parent-process att hålla reda på sina
child-processer är via returvärdet från fork.

\item \emph{UNIX håller flera nivåer av tabeller för öppna filer, både en
  användarspecifik ''File Descriptor Table'' och en global ''File Table''.
Behövs egentligen File Table? Kan man ha offset i File Descriptor Table
istället?}

-- Nej, ''File Table'' behövs. Förutom offsets håller den reda på "access mode"
(t.ex. read/write) samt antalet "File Descriptor Table"-entries som pekar på
den.\\
Denna mellanhand är bland annat användbar för "In-memory inode table", som inte
har dubletter. Bara om en fil vill öppna en för nuvarande stängd fil skapas
ett entry.

\item \emph{Kan man strunta i att stänga en pipe om man inte använder den? Hur
  skulle programbeteendet påverkas? Testa själv.}

-- Nej. Read-änden av pipen får EOF när pipen stängs. Utan detta blockeras
programmet vid läsningen från pipen.
Om pipen inte alls används är det fortfarande ett problem eftersom det är en
oanvänd resurs. Det är alltså en sorts ''memory leak''. När man kör |fork|
skapas det dubletter av parent-processens file descriptors. Detta skapar då
onödiga entries i ''File Descriptor Table''. Man ska inte förvänta sig att
OSet hanterar detta automatiskt.

\item \emph{Vad händer om en av processerna plötsligt dör? Kan den andra
  processen upptäcka detta?}

-- Parent-processen kan få exit-status av sina children via |wait(&status)|.

\item \emph{Hur kan du i ditt program ta reda på om grep misslyckades? Dvs om
  grep inte hittade någon förekomst av det den skulle söka efter eller om du
gett felaktiga parametrar till grep?}

-- Från man-filen:
\begin{verbatim}
Grep will return:
    0     One or more lines were selected.
    1     No lines were selected.
    >1    An error occurred.
\end{verbatim}

Man kan alltså kontrollera "signal"-variabeln efter |wait|.

\end{enumerate}

\section{Problembeskrivning}
Laborationen går ut på att skriva ett litet program som hjälper en att ta reda
på vilka environmentvariabler som är satta.
Programmet ska köras genom att användaren skriver:
\\
|digenv| \emph{parametrar}
\\
Programmet ska internt använda sig av systemkommandona |printenv|, |grep|,
|sort| och |less|.
Med hjälp av nya processer och pipes ska dessa kommmandon kopplas ihop
så att de motsvarar något av:
\\
|printenv |\textbar| sort |\textbar| less| \\ eller \\
|printenv |\textbar| grep | \emph{parametrar} \textbar | sort |\textbar| less |
\\
Om inga argument ges ska det första alternativet köras, om argument ges ska
de skickas till grep.
Vilken pager som används för att visa resultaten bestäms av
environment-variabeln PAGER, om den är definierad.
Om den inte är definierad kommer |less| att användas som basfall.


\section{Programbeskrivning}

Se appendix för exempelkörning.\\

%'' Beskriv hur du valt att lösa uppgiften, speciellt val av datastrukturer och
%programstrukturen på den kod du har skrivit. Motivera också i berörda fall
%varför du har valt att lösa uppgiften på detta sätt. Inkludera en beskrivning
%av de funktioner du skrivit med respektive resultat- och argumentparametrar.
%Beskriv också om och hur du kan ändra beteendet, t. ex. byta ingående
%algoritmer, i ditt program.''

% Detta är bara lite notes.
Alla kommandon som ska köras har fått wrapperfunktioner som exekverar kommandot
i den nuvarande tråden.
Programmet kör sedan sina kommandon i barnprocesser med hjälp av en array av
funktionspekare till dessa wrapperfunktioner.
\\

Funktionen create\_child stegar igenom denna array rekursivt och exekverar
funktionerna i baklänges ordning, så att den process som skapas först dör sist
(FILO). Process-strukturen blir alltså |less|-->|sort|-->|grep|-->|printenv|,
där varje process väntar på att sitt barn ska avlsuta med |waitpid()| innan
kommandot exekveras. Denna design gör att det är enkelt att sätta in nya
processer i pipen, särskilt om dessa inte ska ta emot några argument skilda
från grep-argumenten.
\\ % nåt sånt-ish...

Om någon av dessa processer får en nollskild exit status returneras denna även
av dess parent. Med andra ord kommer processen som skulle köra |less| att få
tillbaka en error om någon av barnprocesserna returnerar error.
Detta innebär att man måste specialhantera greps exit-kod om man vill berätta
för användaren att inga matchningar hittades, eftersom grep signalerar detta
genom att returnera exit-koden 1.
\\ % ska vi skriva sånt här?

Programmet är skrivet så att fyra processer alltid körs. Om grep inte får
några argument ersätts den av en funktion som kallas "pipethrough" som i stort
sett emulerar kommandot |cat| genom att skriva från stdin till stdout.
Detta är inte nödvändigt, men det är utformat så för att enkelt kunna utöka
detta steg med mer funktionalitet och dynamiskt hantera om ett kommando inte
ska köras (som |grep| i vissa fall).


\newpage
\section*{Appendix}
\subsection*{immortal.c:}

\begin{verbatim}
#include <stdio.h>
#include <stdlib.h>
#include <signal.h>
#include <errno.h>
#include <unistd.h>
#include <sys/types.h>

/*
 * Register a signal handler
 */
void register_sighandler( int signal_code, void (*handler) (int sig) )  {
    int retval;

    struct sigaction signal_parameters;

    signal_parameters.sa_handler = handler;
    sigemptyset(&signal_parameters.sa_mask);
    signal_parameters.sa_flags = 0;

    retval = sigaction(signal_code, &signal_parameters, (void *) 0);

    if(-1 == retval) {
        perror("sigaction() failed");
        exit(1);
    }
}

void kill_handler(int signal_code) {
    /* Do nothing */
    printf("%s\n", "kill_handler called");
}

int main() {
    register_sighandler(SIGKILL, kill_handler);
    int a = 0;
    while(1) {
        ++a;
    }
}
\end{verbatim}
\newpage


\subsection*{Exempelkörning:}

\begin{verbatim}
$ gcc -Wall digenv.c -o digenv
$ ./digenv 
Apple_PubSub_Socket_Render=/tmp/launch-HvRHti/Render
CLASSPATH=/Applications/weka-3-6-10/weka.jar:/Applications/weka-3-6-10/weka-src.jar:.
CLICOLOR=1
DISPLAY=/tmp/launch-KHvn2R/org.macosforge.xquartz:0
GOROOT=/Users/mrunelov/Programming/mygo
HOME=/Users/mrunelov
LANG=sv_SE.UTF-8
LOGNAME=mrunelov
LSCOLORS=GxFxCxDxBxegedabagaced
OLDPWD=/Users/mrunelov/GitHub/MGM-KTH
PATH=/usr/local/Cellar/qt/4.8.5/bin:/usr/local/mysql/bin:/Users/mrunelov/scripts:/opt/local/bin:/opt/local/sbin:/usr/local/bin:/usr/local/google_appengine:/usr/local:/Library/Frameworks/Python.framework/Versions/2.7/bin:/usr/local/bin:/usr/bin:/bin:/usr/sbin:/sbin:/opt/X11/bin:/usr/local/go/bin:/usr/texbin:/Users/mrunelov/Programming/mygo/bin:/usr/local/go/bin
PS1=\W [\u]$ 
PWD=/Users/mrunelov/GitHub/MGM-KTH/pipe
PYTHONPATH=/usr/local/Cellar/qt/4.8.5/bin:/usr/local/lib/python2.7/site-packages:
SHELL=/usr/local/bin/bash
SHLVL=1
SSH_AUTH_SOCK=/tmp/launch-N6JjM3/Listeners
TERM=xterm-256color
TERM_PROGRAM=Apple_Terminal
TERM_PROGRAM_VERSION=326
TERM_SESSION_ID=39F6CA28-13D0-4965-AFDD-B46D678C9F0C
TMPDIR=/var/folders/pt/xv7svsgx3_3gw16czfy_zx9r0000gn/T/
USER=mrunelov
_=./digenv
__CF_USER_TEXT_ENCODING=0x1F5:0:0
__CHECKFIX1436934=1
(END) 
$ ./digenv SHELL
SHELL=/usr/local/bin/bash
(END) 
$ ./digenv -c SH*
13
(END)
\end{verbatim}

%------------------------------------------------------------------------------

\end{document}
