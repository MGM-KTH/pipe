%%%%%%%%%%%%%%%%%%%%%%%%%%%%%%%%%%%%%%%%%
% Short Sectioned Assignment
% LaTeX Template
% Version 1.0 (5/5/12)
%
% This template has been downloaded from:
% http://www.LaTeXTemplates.com
%
% Original author:
% Frits Wenneker (http://www.howtotex.com)
%
% License:
% CC BY-NC-SA 3.0 (http://creativecommons.org/licenses/by-nc-sa/3.0/)
%
%%%%%%%%%%%%%%%%%%%%%%%%%%%%%%%%%%%%%%%%%

%----------------------------------------------------------------------------------------
%	PACKAGES AND OTHER DOCUMENT CONFIGURATIONS
%----------------------------------------------------------------------------------------

\documentclass[paper=a4, fontsize=11pt]{scrartcl} % A4 paper and 11pt font size

\usepackage[T1]{fontenc} % Use 8-bit encoding that has 256 glyphs
%\usepackage{fourier} % Use the Adobe Utopia font for the document - comment this line to return to the LaTeX default
\usepackage[english]{babel} % English language/hyphenation
\usepackage{amsmath,amsfonts,amsthm} % Math packages

\usepackage[utf8]{inputenc} % Needed to support swedish "åäö" chars
\usepackage{titling} % Used to re-style maketitle
\usepackage{enumerate}
\usepackage{lipsum} % Used for inserting dummy 'Lorem ipsum' text into the template

\usepackage{sectsty} % Allows customizing section commands
\allsectionsfont{\normalfont} % Make all sections centered, the default font and small caps

\usepackage{fancyhdr} % Custom headers and footers
\pagestyle{fancyplain} % Makes all pages in the document conform to the custom headers and footers
\fancyhead{} % No page header - if you want one, create it in the same way as the footers below
\fancyfoot[L]{} % Empty left footer
\fancyfoot[C]{} % Empty center footer
\fancyfoot[R]{\thepage} % Page numbering for right footer
\renewcommand{\headrulewidth}{0pt} % Remove header underlines
\renewcommand{\footrulewidth}{0pt} % Remove footer underlines
\setlength{\headheight}{13.6pt} % Customize the height of the header

\numberwithin{equation}{section} % Number equations within sections (i.e. 1.1, 1.2, 2.1, 2.2 instead of 1, 2, 3, 4)
\numberwithin{figure}{section} % Number figures within sections (i.e. 1.1, 1.2, 2.1, 2.2 instead of 1, 2, 3, 4)
\numberwithin{table}{section} % Number tables within sections (i.e. 1.1, 1.2, 2.1, 2.2 instead of 1, 2, 3, 4)

\setlength\parindent{0pt} % Removes all indentation from paragraphs - comment this line for an assignment with lots of text

\usepackage{fancyvrb}
\DefineShortVerb{\|}


\posttitle{\par\end{center}} % Remove space between author and title
%----------------------------------------------------------------------------------------
%	TITLE SECTION
%----------------------------------------------------------------------------------------

\begin{document}

\section{Problembeskrivning}
Laborationen går ut på att skriva ett litet program som hjälper en att ta reda
på vilka environmentvariabler som är satta.
Programmet ska använda sig av systemkommandona |printenv|, |grep|, |sort|
och |less|.
Med hjälp av nya processer och pipes ska dessa kommmandon kopplas ihop
så att de motsvarar något av:
\\
|printenv |\textbar| sort |\textbar| less| \\ eller \\
|printenv |\textbar| grep | \emph{parametrar} | sort |\textbar| less | \\


\section{Förberedelsefrågor}
\begin{enumerate}[1)]
\item \emph{När en maskin bootar med UNIX skapas en process som har PID=1 och den lever så länge maskinen är uppe. Från den här processen skapas alla andra processer med fork. Vad heter denna process?}

-- Processen heter \emph{init} på Linux, launchd på Mac.

\item \emph{Kan environmentvariabler användas för att kommunicera mellan föräldra- och barnprocess? åt bägge hållen?}

-- Child-processer får en kopia av parent-processens environmentvariabler. Detta innebär att en parent kan bestämma startvärdena för children.
Eftersom child-processen får en kopia speglas inte ändringar i denna i parent-processens variabler. Alltså kan inte child-processer kommunicera till parent via dessa.

\item \emph{Man kan tänka sig att skapa en odödlig child-process som fångar alla SIGKILL-signaler genom att registrera en egen signalhanterare kill\_handler som bara struntar i SIGKILL. Processen ska förstås ligga i en oändlig loop då den inte exekverar signalhanteraren. Testa! Skriv ett program med en sådan signalhanterare, kompilera och provkör. Vad händer?}

-- Se appendix för programmet, |immortal.c| \\
-- Det är inte tillåtet att fånga SIGKILL då detta skulle vara en säkerhetsrisk. Det går däremot att fånga SIGINT.
Resultat av ett program som försöker registrera en signalhanterare för SIGKILL:

\begin{verbatim}
sigaction() failed: Invalid argument
\end{verbatim}

Förklaringen finns i man-filen för sigaction:
\begin{verbatim}
ERRORS
     The sigaction() system call will fail and no new signal handler will be
     installed if one of the following occurs:

     [EINVAL]           An attempt is made to ignore or supply a handler for
                        SIGKILL or SIGSTOP.
\end{verbatim}

\item \emph{Varför returnerar fork 0 till child-processen och child-PID till parent-processen, i stället för tvärtom?}


\item \emph{UNIX håller flera nivåer av tabeller för öppna filer, både en användarspecifik “File Descriptor Table” och en global “File Table”. Behövs egentligen File Table? Kan man ha offset i File Descriptor Table istället?}


\item \emph{Kan man strunta i att stänga en pipe om man inte använder den? Hur skulle programbeteendet påverkas? Testa själv.}


\item \emph{Vad händer om en av processerna plötsligt dör? Kan den andra processen upptäcka detta?}


\item \emph{Hur kan du i ditt program ta reda på om grep misslyckades? Dvs om grep inte hittade någon förekomst av det den skulle söka efter eller om du gett felaktiga parametrar till grep?}

\end{enumerate}

\section{Programbeskrivning}

\newpage
\section*{Appendix}
immortal.c:\\

\begin{verbatim}
#include <stdio.h>
#include <stdlib.h>
#include <signal.h>
#include <errno.h>
#include <unistd.h>
#include <sys/types.h>

/*
 * Register a signal handler
 */
void register_sighandler( int signal_code, void (*handler) (int sig) )  {
    int retval;

    struct sigaction signal_parameters;

    signal_parameters.sa_handler = handler;
    sigemptyset(&signal_parameters.sa_mask);
    signal_parameters.sa_flags = 0;

    retval = sigaction(signal_code, &signal_parameters, (void *) 0);

    if(-1 == retval) {
        perror("sigaction() failed");
        exit(1);
    }
}

void kill_handler(int signal_code) {
    /* Do nothing */
    printf("%s\n", "kill_handler called");
}

int main() {
    register_sighandler(SIGKILL, kill_handler);
    int a = 0;
    while(1) {
        ++a;
    }
}
\end{verbatim}
\newpage


%----------------------------------------------------------------------------------------

\end{document}
